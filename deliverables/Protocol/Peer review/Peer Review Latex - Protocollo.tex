\documentclass[12pt]{article}
\usepackage[utf8]{inputenc}
\usepackage[T1]{fontenc}
\usepackage[italian]{babel}

\title{Peer-Review 2: Protocollo di Comunicazione}
\author{Riccardo Mazza, Greta Gaia Moneta, Luca Pizzocolo\\Gruppo 10}

\begin{document}

\maketitle

Valutazione della documentazione del protocollo di comunicazione del gruppo 9.

\section{Lati positivi}
\subsection{Join}
Guardando il sequence diagram della Join abbiamo trovato solo lati positivi: il diagramma è coerente con l'entrata dei giocatori, ci è sembrato completo.
\subsection{Planning Phase}
Anche in questo caso la sequenza di azioni è lineare, però, leggendo la documentazione del protocollo, crediamo che non si riesca completamente a comprendere come avvenga effettivamente l'update della view.
\subsection{Action Phase 1/2/3}
Abbiamo deciso di mettere tutte le action phase in un'unica sezione poichè i 3 sequence diagram sono molto simili il che ci sembra positivo poichè sono facili da comprendere in poco tempo. Al contempo però il fatto che le sequenze di azioni manchino di dettaglio risulta negativo dato che non riusciamo a giudicare fino in fondo. 
\subsection{EndGame}
In questo caso il diagramma ci sembra concettualmente corretto ma, come detto prima, non essendo sufficientemente esaustivo, facciamo fatica a dare la nostra opinione. In ogni caso guardando il sequence diagram ci sembra non manchino i messaggi della fase.

\section{Lati negativi}
\subsection{Planning Phase}
Analizzando il diagramma e la documentazione del protocollo troviamo che non sia chiaro come avviene effettivamente l'update della view.
\subsection{Action Phase 1/2/3}
Non è intuibile cosa sia la 'move' (Si intende un oggetto? O una Stringa?), quindi non possiamo sapere cosa sia e di conseguenza non riusciamo a dare un giudizio 
\subsection{EndGame}
La nostra idea di sequence diagram risulta molto più dettagliata e, leggendo questo tipo di diagramma (più breve), sembra sia meglio per la velocità in cui si legge, ma pensiamo che non sia sufficientemente chiaro per giudicare completamente il protocollo di rete. Ad esempio in questa sezione non riusciamo a comprendere cosa sia la' 'update'.

\section{Confronto}
\subsection{Join}
Una differenza di implementazione che c'è tra i nostri progetti è che noi abbiamo deciso di creare le partite multiple, così da avere la scelta del numero dei giocatori e della modalità di gioco.
\subsection{Planning Phase}
Abbiamo deciso di prendere come spunto dal sequence diagram il tempo per ogni mossa, oltre al fatto che sia casuale in caso di scadenza del timeout.
\subsection{Action Phase 1/2/3}
Abbiamo preferito evitare i loop per le mosse anche se tecnicamente non ci sembra errato, noi preferiamo dare un tempo generale alla fase.
\subsection{EndGame}
L'unico confronto per la fine delle partita che abbiamo trovato è il fatto che noi abbiamo deciso di mandare messaggi sia di vittoria che di sconfitta.

\end{document}
